\documentclass[letterpaper,12pt]{article}
\usepackage{tabularx} % extra features for tabular environment
\usepackage{amsmath}  % improve math presentation
\usepackage{graphicx} % takes care of graphic including machinery
\usepackage[margin=1in,letterpaper]{geometry} % decreases margins
\usepackage{cite} % takes care of citations
\usepackage[final]{hyperref} % adds hyper links inside the generated pdf file
\hypersetup{
	colorlinks=true,       % false: boxed links; true: colored links
	linkcolor=blue,        % color of internal links
	citecolor=blue,        % color of links to bibliography
	filecolor=magenta,     % color of file links
	urlcolor=blue         
}
%++++++++++++++++++++++++++++++++++++++++
\begin{document}
\title{ Computer vision \protect\\ Project report \protect\\ Recognition based localization}
\author{Baert Olivier, Callaert Arne, Hernou Arne, Vereecken Ward, Waegemans Wolfgang}
% \date{9 februari 2021}
\maketitle

\begin{abstract}
In this project, a program is created with the goal to localize a person based on captured video footage. To achieve this, first painting are detected in the footage frame by frame. These detected paintings are then matched with paintings in a database containing all the museum's paintings. Based on this matching, the person is localized on a map of the museum.
\end{abstract}

\section{Assignment 1: painting detection}

In this assignment, the focus lies on detecting any shape in a given image that is most likely to be a painting, completely \textbf{unsupervised} (without any human intervention). 

\section{Assignment 2: painting matching}

\section{Assignment 3: localization}

\section{Assignment 4: demo}

%\begin{figure}
	%\centering
	%\centerline{
%			\includegraphics[scale=0.47]{clouds}
	%}
	%\caption{\textit{clouds.png}, de input afbeelding}
%\end{figure}
%\label{fig:clouds-input}

\end{document}